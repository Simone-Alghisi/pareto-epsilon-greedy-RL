\IEEEPARstart{P}{okémon} is a popular turn-based game where two players must choose a monster move and a target each turn with the aim of knocking out all the opponent's Pokémons. The game is far from being trivial and has many different possible strategies, leading to a very high-dimensional search space. Moreover, each of the player's Pokémon has unique statistics\footnote{\url{https://bulbapedia.bulbagarden.net/wiki/Stat\#List_of_stats}} such as Speed, HP (Health Points), Attack, and Defence. The fact that the opponent Pokémons also have their own unknown parameters makes the game even more convoluted. In particular, there are $10^{354}$ different ways a Pokémon battle can start, and each turn has at most $306$ different outcomes (and only for a single player). The game's complexity turns out to be a problem for the development of a Pokémon AI, as high complexity implies a very large number of possible outcomes to choose from. Since the analysis of all possibilities is not feasible, it is imperative to implement a strategy to efficiently find an ideal move. At the beginning of a \emph{Reinforcement Learning (RL)} training procedure, an agent does not know which action is the most effective in order to win the battle. The purpose of this study is to give the Pokémon AI agent a set of potentially useful moves to warm up the training process and have a faster training convergence. Since each move has a different effect on the game and it is driven by the dynamics of the combat, the term "useful move" is ambiguous and there is no one-size-fits-all approach. Genetic Algorithms, in particular \emph{NSGA-II}, allow to produce \emph{Pareto-equivalent} solutions of a multi-objective optimisation problem, which is suitable in order to find the best move a Pokémon can perform.